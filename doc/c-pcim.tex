\documentclass[12pt]{article}

\usepackage[top=1.00in, bottom=1.25in, left=0.75in, right=0.75in]{geometry}
\usepackage{amsmath, amssymb, amsthm}
\usepackage[round]{natbib}
\usepackage{url}
\usepackage{breakurl}
\usepackage[breaklinks]{hyperref}
\def\UrlBreaks{\do\/\do-}

\theoremstyle{plain}
\newtheorem*{theorem}{Theorem}
\newtheorem*{lemma}{Lemma}
\newtheorem*{corollary}{Corollary}

\theoremstyle{definition}
\newtheorem{definition}{Definition}
\newtheorem*{convention}{Convention}

\theoremstyle{remark}
\newtheorem*{notation}{Notation}
\newtheorem*{note}{Note}
\newtheorem*{claim}{Claim}

\newcommand{\suchthat}{\mid}
\newcommand{\given}{\mid}


\begin{document}

\begin{definition}
  Discrete data is counted while continuous data is measured
  \citep{matisfundis18}.
\end{definition}

\begin{definition}
  A \textbf{combination} is an arrangement of items without regard to the order.
  A \textbf{permutation} is an arrangement of items in a specific order.
  \citep{matisfuncom18}.
\end{definition}

\begin{definition}
  An \textbf{experiment} is a procedure that yields one of a given set of
  possible outcomes \citep{ros02}.
\end{definition}

\begin{definition}
  The \textbf{sample space} of the experiment is the set of possible outcomes
  \citep{ros02}.
\end{definition}

\begin{definition}
  An \textbf{event} is a subset of the sample space \citep{ros02}.
\end{definition}

\begin{definition}
  \textbf{Probability} is the measure of the likelihood that an event will
  occur.
  Probability is quantified as a number between 0 and 1, where 0 indicates
  impossibility and 1 certainty.
\end{definition}

\begin{definition}
  \textbf{Finite probability}. If $S$ is a finite nonempty sample space of
  equally likely outcomes, and $E$ is an event, that is, a subset of $S$,
  then the probability of $E$ is $p(E)=\frac{|E|}{|S|}$ \citep{ros02}.
\end{definition}

\begin{definition}
  A \textbf{random variable}\footnotemark\ is a function from the sample space
  of an experiment to the set of real numbers.
  \begin{equation*}
    X:S \to \mathbb{R}
  \end{equation*}
  A random variable $X(t)$ maps each outcome $t$ to a real number \citep{ros02}.
  \footnotetext{
    Random variables translate outcomes in the sample space of an experiment to
    real numbers so that we can reason about them with mathematics.}
\end{definition}

\begin{note}
  It is worth clarifying that ``random variable'' is a misnomer.
  A random variable is a function.
  It is not a variable, and it is not random!
  It is a representation of an underlying random system.
  The name \textit{random variable} (the translation of
  \textit{variabile casuale}) was introduced by the Italian mathematician
  F. P. Cantelli in 1916 \citep{ros02}.
\end{note}

\section{Probability Distributions}

\begin{definition}
  The \textbf{distribution} of a random variable $X$ on a sample space $S$
  is the set of pairs $(r,p(X=r))$ for all $r \in X(S)$, where $p(X=r)$ is the
  probability that $X$ takes the value $r$.
  The set of pairs in this distribution is determined by the probabilities
  $p(X=r)$ for $r \in X(S)$ \citep{ros02}.
\end{definition}

\subsection{Bernoulli Distribution}

\begin{definition}
  Discrete probability distribution of a random variable which takes the value
  $1$ with probability $p$ and the value $0$ with probability $q=1-p$.
\end{definition}

The random variable:
\begin{equation*}
  X = \text{The outcome of the experiment is ``success''.}
\end{equation*}

Probability mass function over possible outcomes $k$:
\begin{equation*}\label{bernoulli}
  f(k;p) = \mathrm{Pr}(k;p) = \mathrm{Pr}(X=k) = p^k(1 - p)^{1-k}
  \quad \text{for}\ k \in \{0,1\}
\end{equation*}

\subsection{Binomial Distribution}

\begin{definition}
  Discrete probability distribution of the number of successes in a sequence of
  $n$ independent Bernoulli experiments.
  For a single trial, that is, $n=1$, the binomial distribution is a Bernoulli
  distribution.
\end{definition}

\begin{note}
  The binomial distribution is frequently used to model the number of successes
  in a sample of size $n$ drawn with replacement from a population of size
  $N$.
  If the sampling is carried out without replacement, the draws are not
  independent and so the resulting distribution is a \textbf{hypergeometric}
  distribution, not a binomial one. However, for $N$ much larger than $n$, the
  binomial distribution remains a good approximation, and is widely used.
\end{note}

The random variable:
\begin{equation*}
  X = \text{The number of successes in a sequence of $n$ independent Bernoulli
            experiments is $k$.}
\end{equation*}

Probability mass function of getting exactly $k$ successes in $n$ trials:
\begin{equation*}%\label{binomial}
  f(k;n,p) = \mathrm{Pr}(k;n,p) = \mathrm{Pr}(X=k) = \binom{n}{k}p^k(1-p)^{n-k}
  \quad \text{for}\ k \in \{0,1,\dots,n\}
\end{equation*}

\begin{definition}
  The binomial coefficient, $\binom{n}{k} = \frac{n!}{k!(n-k)!}$, is the
  coefficient of the $x^k$ term in the polynomial expansion of the binomial
  power
  $(1+x)^n = \binom{n}{0}x^0 + \binom{n}{1}x^1 + \cdots + \binom{n}{n}x^n$,
  that is, the \textbf{binomial theorem}.
\end{definition}

\paragraph{Explanation}
$k$ successes occur with probability $p^k$ and $n-k$ failures occur with
probability $(1-p)^{n-k}$.
The $k$ successes can occur anywhere among the $n$ trials, and there are
$\binom{n}{k}$ different ways of distributing $k$ successes in a sequence of
$n$ trials.

\subsection{Poisson Distribution}

\begin{definition}
  Discrete probability distribution of a given number of events occurring in a
  fixed space interval, usually time, and these events occur with a known
  constant rate and independently of the time since the last event.
  The Poisson distribution is derived from the \textbf{binomial distribution}
  by splitting up the interval into $n$ subintervals, each of which is so small
  that at most one event could occur in it with non-zero probability
  \citep{wacmensch08}.
\end{definition}

The random variable:
\begin{equation*}
  X = \text{The number of events in an interval is $k$.}
\end{equation*}

Probability mass function\footnotemark\ of getting exactly $k$ events in an
interval:
\footnotetext{
  Notice that the interval length is not specified.
  Refer to \citet{wacmensch08} for the derivation.
  }
\begin{equation*}
  f(k;\lambda) = \mathrm{Pr}(k;\lambda) = \mathrm{Pr}(X=k) =
  e^{-\lambda}\frac{\lambda^k}{k!}
\end{equation*}

where $\lambda$ is the event rate, the average number of events in an interval.

\subsection{Exponential Distribution}

\begin{definition}
  The exponential distribution (also known as negative exponential distribution)
  is the probability distribution that describes the time between events in a
  Poisson point process, i.e., a process in which events occur continuously and
  independently at a constant average rate.
  It is a particular case of the gamma distribution.
  It is the continuous analogue of the geometric distribution, and it has the
  key property of being memoryless.
  In addition to being used for the analysis of Poisson point processes it is
  found in various other contexts.
\end{definition}

The random variable (continuous):
\begin{equation*}
  X = \text{The amount of time that passes before the next event.}
\end{equation*}

Probability density function:
\begin{equation*}
  f(x;\lambda) = \lambda e^{-\lambda x}\quad \text{for}\ x \geq 0
\end{equation*}
where $x$ represents time, hence it is defined only for $x \geq 0$.

\subsection{Stochastic (Random) Process}

\begin{definition}
  A \textbf{stochastic process} is a random phenomenon that arises through a
  process which is developing in time in a manner controlled by probabilistic
  laws \citep{par99}.
  From a point of view of the mathematical theory of probability a stochastic
  process is best defined as a collection of random variables, indexed by points
  in some space.
  \begin{equation*}
    \{X(t), t \in T\}
  \end{equation*}
  A more appropriate name in mathematics is \textbf{random field}.
  The set used to index the random variables is called the \textbf{index set}.
  Historically, the index set was some subset of the real line, such as the
  natural numbers, giving the index set the interpretation of time.
  Each random variable in the collection takes values from the same mathematical
  space known as the \textbf{state space}.
  An \textbf{increment} is the amount that a stochastic process changes between
  two index values, often interpreted as two points in time.
\end{definition}

\begin{note}
  Even though the index set can be any set in any space, generally more results
  and theorems are possible  for stochastic processes when the index set is
  ordered.
  Most commonly, random variables in a stochastic process are indexed by the
  positive numbers along the real number line, interpreted as time.
  Among these, the \textbf{Brownian motion process} and
  the \textbf{Poisson process} are considered the most important and central in
  the theory of stochastic processes.
  Other stochastic processes include \textbf{random walk} and
  \textbf{Markov Chain} and \textbf{martingales}.
  A martingale models a fair game.
  A simple random walk is a Markov chain and also a martingale.
\end{note}

\subsection{Point Process}

\begin{definition}
  A point process (or \textbf{point field}) is a collection of mathematical
  points randomly located on some underlying mathematical space.
  Point processes on the real line form an important special case that is
  particularly amenable to study because the points are ordered in a natural
  way, and the whole point process can be described completely by the (random)
  intervals between the points.
\end{definition}

\begin{note}
  Different from stochastic processes, point processes are defined as a
  collection points randomly located on some space and not as a collection of
  random variables indexed by some index space.
  Various interpretations of a point process exist:
  \begin{itemize}
  \item a random object that arises from or is associated with a stochastic
    process
  \item a stochastic process indexed by sets of the underlying space on which it
    is defined
  \end{itemize}
\end{note}

\subsection{Renewal (Point) Process}

\begin{definition}
  Historically the first point processes that were studied had the real half
  line $\mathbb{R}_+ = [0,\infty)$ as their state space, which in this context
  is usually interpreted as time.
  These studies were motivated by the wish to model telecommunication systems,
  in which the points represented events in time, such as calls to a telephone
  exchange.
  Point processes on $\mathbb{R}_+$ are typically described by giving the
  sequence of their (random) inter-event times $(T1,T2,\dots)$, from which the
  actual sequence $(X1,X2,\dots)$ of event times can be obtained as
  \begin{equation*}
    X_k = \sum_{j=1}^{k} T_j\quad \text{for}\ k \geq 1.
  \end{equation*}
  If inter-event times are independent and identically distributed, the point
  process is called a \textbf{renewal process}.
\end{definition}

\begin{note}
  The \textbf{intensity} $\lambda(t|H_t)$ of a point process on the real
  half-line with respect to a filtration $H_t$ is defined as
  \begin{equation*}
    \lambda(t|H_t) = \lim_{\Delta t \to 0} \frac{1}{\Delta t} \mathrm{Pr}
    (\text{One event occurs in time interval} [t, t+\Delta t] | H_t)
  \end{equation*}
  $H_t$ can denote the history of event-point times preceding time $t$ but can
  also correspond to other \emph{filtrations} (for example in the case of a Cox
  process).
  The \emph{compensator} of a point process, also known as the dual-predictable
  projection, is the integrated conditional intensity function defined by
  \begin{equation*}
    \Lambda(s,u) = \int_s^u \lambda(t|H_t)\mathrm{d}t
  \end{equation*}
\end{note}

% \footnote is a fragile command and section (or subsection) title are moving
% arguments (meaning they get written to an auxiliary file to be used in the
% table of contents). Fragile commands break in moving arguments and the
% standard solution is to protect them with \protect. However, this will produce
% a footnote in the table of contents. In order to avoid this, use the optional
% argument of \section for the TOC or even better, do not use footnotes in
% headings.
% https://tex.stackexchange.com/questions/153329/footnote-in-sub-section-title
\subsection[Poisson Process]{Poisson (Point)\footnotemark\ Process}

\footnotetext{
  The word point is often omitted, but there are other Poisson processes of
  objects, which, instead of points, consist of more complicated mathematical
  objects such as lines and polygons, and such processes can be based on the
  Poisson point process.
}

\begin{definition}
  The number of points in a region of finite size within a Poisson process is
  a random variable with a \textbf{Poisson distribution}.
  The Poisson point process is often defined on the real line, where it can be
  considered as a \textbf{stochastic process}.
  The Poisson process is a point process with convenient mathematical
  properties.
  The Poisson point process has the property that each point is stochastically
  independent to all the other points in the process.
  The Poisson process has been used to build other point processes where the
  points are not independent of each other.
\end{definition}

\begin{note}

  \textbf{Poisson Counting Process}

\begin{definition}
  A homogeneous Poisson point process on the positive half-line that represents
  the total number of occurrences or events that have happened up to and
  including time $t$.
  \begin{equation*}
    \{ N(t), t \geq 0 \}
  \end{equation*}
  The probability of random variable $N(t)$ being equal to $n$:
  \begin{equation*}
    P\{N(t)=n\} = e^{\lambda t} \frac{(\lambda t)^n}{n!}
  \end{equation*}
\end{definition}

\end{note}

\pagebreak

\subsection{PCIM}

\begin{definition}
  Model for learning nonlinear temporal dependencies in event streams.
  Events carry both information about their timing and their type.
  Used to predict users' interests from their web queries and the type of system
  from supercomputer system error log.
  The dependencies between events can be due to both their timing and their
  types.

  The \textbf{Piecewise-Constant Conditional Intensity Model} (PCIM) is a class
  of marked point processes that can model the types and timing of events.
  This model captures the dependencies of each type of event on events in the
  past through a set of piecewise-constant conditional intensity functions.
  Decision trees represent these dependencies. The decision trees are the
  picewise-constant conditional intensity functions \citep{gunmeexu11}

  A conjugate prior for this model allows for closed-form computation of the
  marginal likelihood and parameter posteriors.
  Model selection then becomes a problem of choosing a decision tree.
  Decision tree induction can be done efficiently because of the closed form for
  the marginal likelihood.

  \textbf{Poisson Networks} \citep{rajgraher05} (also piece-wise constant
  conditional intensity models) are closely related to PCIMs.

\textbf{Conditional Intensity Model} \citep{did08,dalver09}:
\begin{equation*}
  p(x|\theta) = \prod_{l \in \mathcal{L}} \prod_{i=1}^{n}
  \lambda_l(t_i|h_i,\theta)^{\mathbf{1}_l(l_i)}
  e^{-\Lambda_l(t_i|h_i;\theta)}
\end{equation*}
where
$\Lambda_l(t|x;\theta) = \int_{-\infty}^t\lambda_l(\tau|x;\theta)\mathrm{d}t$
for each event sequence $x$ and the indicator function $\mathbf{1}_l(l')$ is one if
$l'=l$ and zero otherwise.
The rate parameter changes as a function of time and history and is computed by
the conditional intensity function $\lambda_l(t_i|h_i,\theta)$.

\begin{note}
  The \emph{Conditional Intensity Model} is analogous to the likelihood of the
  sequence of samples from the exponential distribution:
  Let $X$ be exponential distributed with rate parameter $\lambda$ and
  $x_1,\dots,x_n$ are $n$ independent samples from $X$.
  The likelihood function is given by:
  \begin{equation*}
    L(\lambda;x) = p(x;\lambda) = \prod_{i=1}^n\lambda e^{-\lambda x_i}
  \end{equation*}
  When there is only one label, the conditional intensity model becomes
  \begin{equation*}
    p(x|\theta) = \prod_{i=1}^n\lambda(t_i|h_i,\theta)
    e^{-\Lambda(t_i|h_i;\theta)}
  \end{equation*}
  and when the rate parameter is fixed $\lambda(t_i|h_i,\theta) = \lambda$ it
  becomes $L(\lambda;x)$.
\end{note}

\end{definition}

\bibliography{bibliography}
\bibliographystyle{plainnat}
\end{document}

%%% Local Variables:
%%% mode: latex
%%% TeX-master: t
%%% End:
